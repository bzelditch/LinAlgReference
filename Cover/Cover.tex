\documentclass[paper=a4, fontsize=11pt,twoside]{scrartcl}   % KOMA

\usepackage[a4paper,pdftex]{geometry}   % A4paper margins
\setlength{\oddsidemargin}{5mm}         % Remove 'twosided' indentation
\setlength{\evensidemargin}{5mm}

\usepackage[english]{babel}
\usepackage[protrusion=true,expansion=true]{microtype}  
\usepackage{amsmath,amsfonts,amsthm,amssymb}
\usepackage{graphicx}

% --------------------------------------------------------------------
% Definitions (do not change this)
% --------------------------------------------------------------------
\newcommand{\HRule}[1]{\rule{\linewidth}{#1}}   % Horizontal rule

\makeatletter                           % Title
\def\printtitle{%                       
    {\centering \@title\par}}
\makeatother                                    

\makeatletter                           % Author
\def\printauthor{%                  
    {\centering \large \@author}}               
\makeatother                            

% --------------------------------------------------------------------
% Metadata (Change this)
% --------------------------------------------------------------------
\title{  
                  % [.5cm]   
          % \HRule{0.5pt} \\                      % Upper rule
            \LARGE \textbf{\uppercase{Linear Algebra}}    % Title
           %\HRule{2pt} \\ [0.5cm]      % Lower rule + 0.5cm spacing
                    % Todays date
}
        


\author{
        Ben Zelditch\\   
        Cornell University\\  
        \texttt{bz87@cornell.edu} \\
}


\begin{document}
% ------------------------------------------------------------------------------
% Maketitle
% ------------------------------------------------------------------------------
\thispagestyle{empty}       % Remove page numbering on this page

\printtitle                 % Print the title data as defined above
    \vfill
\printauthor                % Print the author data as defined above
\newpage
% ------------------------------------------------------------------------------
% Begin document
% ------------------------------------------------------------------------------
      % Set page numbering to begin on this page

 \textit{He who controls the past controls the future. He who controls the present controls the past.} 
 
 \vskip 1mm
		- George Orwell
\section{Personal Background}

I have always had an uneasy relationship with linear algebra. The first time I encountered this subject was
the second semester of my senior year of high school. Having already been accepted into college, I wasn't exactly
a star student in any of my classes and linear algebra was no exception. The concept of a linear transformation between two
vector spaces, let alone a vector space in its own right, was totally lost on me. My only takeaway from the class was my memorizing the formula that
computes the determinant of an $n \times n$ matrix. 

Fast forward
to my first semester at Cornell. An overzealous freshman, I enrolled in Math 2230, the theoretical linear algebra and calculus course designed
for freshman math majors. Thrown right into to the world of abstraction without ever having been introduced to even the most basic
mathematical proof, I quickly realized I was out of my element. Freshman stubbornness and an unchecked ego got the best of me, however,
and I ended up sticking with the class for the semester. Achieving an average of 40\% for my exam grades and being saved only by the generous
curve, I left Cornell that semester demoralized. Linear algebra was still a blur.

Then second semester of sophomore year rolled around. Having taken a few more reasonable math classes, my self-confidence was restored. I enrolled in Math 4310, 
a junior-senior level course in linear algebra, with the intention of really learning the subject once and for all. It proved to be a positive experience:  the course was at the right level
of difficulty for me and I enjoyed studying the subject. Having put in substantial effort into the course, I walked away from Cornell that semester satisfied with what I had learned.

Fast forward to today. While I am still confident in my understanding of the major concepts in linear algebra, I can't help but feel a little bit rusty. My intention for this book (book?)
is for it to serve as a linear algebra refresher, at roughly the same level as Math 4310. My hope is that it will serve as a useful reference when I need to apply concepts from this subject to other problems in the future as well. Of course, I hope that others who read this reference can find it similarly useful!

\section{Motivation for Writing This}

\section{What to Expect}



\end{document}