\documentclass[12pt]{article}
\usepackage{mathtools}
\usepackage{amsfonts}
\usepackage{enumerate}
\usepackage{amsthm}
\usepackage{epic}
\usepackage{eepic}
\usepackage{paralist}
\usepackage{graphicx}
\usepackage{algorithm,algorithmic}
\usepackage{tikz}
\usepackage{xcolor,colortbl}
\usepackage{wrapfig}

\title{Chapter 3}
\author{Ben Zelditch}

\begin{document}

\maketitle

\section*{Eigenvalues and Eigenvectors}

In the last chapter we learned all about linear transformations from one vector space to another. Now we're going to focus our attention on the study
of linear operators, which are linear transformations from a vector space to itself. The deepest results in linear algebra are related to linear operators, so this chapter
is going to be a good one. Stay tuned.

\subsection*{Invariant Subspaces}

Before we can think about eigenvalues and eigenvectors, we need a refresher on linear operators. Remember from last chapter that a linear operator $T \in \mathcal{L} (V)$ on a vector space $V$ is a linear map from $V$ to itself. Now imagine that $T$ is some particularly nasty linear map and is complicated to visualize because $V$ is a high dimensional vector space. Remember from Chapter 1 that we could decompose

$$V = U_1 \oplus U_2 \oplus \dots \oplus U_n$$

where $\oplus$ is the direct sum. If we could understand how $T$ behaves on each subspace $U_j$ individually, then we would understand how it behaves on $V$ by linearity. Let $T|_{U_j}$ denote the \textit{restriction} of $T$ to the subspace $U_j$, i.e. we're considering $U_j$ to be the domain of $T$ instead of $V$. Since $U_j$ is smaller than $V$ it makes sense to think that it would easier to deal with $T$ on $U_j$ than on $V$.

But what if $T|_{U_j}$ doesn't map to $U_j$ to itself, i.e. what if there exists some $u \in U_j$ such that $Tu \notin U_j$? In other words, what if $T|_{U_j}$ is not an operator on $U_j$?

Well that would screw things up. Instead we'll only consider decompositions of $V$ such that $T|_{U_j}$ is an operator on $U_j$. This has a name:  we say that a subspace $U \subset V$ is \textit{invariant} under a linear operator $T$ if for all $u \in U$ we have $Tu \in U$, equivalently if $T|_U$ is an operator on $U$. So from now on, we'll only consider direct sum decompositions of invariant subspaces.

Let's drive this point home by looking at a few examples of invariant subspaces. Let $T \in \mathcal{L}(V)$. Then $\{0\}$ and $V$ itself are invariant subspaces (the so called trivial cases). Additionally, $\textrm{null} \: T$ and $\textrm{range} \: T$ are both invariant under $T$ (the proofs for both cases are easy). For a more concrete example, suppose that $T \in \mathcal{L} (\mathbb{R}^2)$ is given by the map

\[ \begin{pmatrix}
x_1 \\
x_2
\end{pmatrix}
%
\mapsto
\begin{pmatrix}
3x_1 \\
x_2
\end{pmatrix}
\]

Then the $x$-axis is invariant under $T$ because if $(x_1, 0)^T$ is a point on the $x$-axis then $T((x_1, 0)^T) = (3x_1, 0)^T$ is also a point on the $x$-axis.

That's about it for invariant subspaces. We'll see why they're so important in the next few sections.


\subsection*{Eigenvalues and Eigenvectors}

Now that we're pretty familiar with invariant subspaces, let's look at a special case of them:  one-dimensional invariant subspaces. Let's first define the one-dimensional subspace.
Let $V$ be a vector space, say $V = \mathbb{R}^n$, and let $u \in V$. Then the set

$$U = \{ \alpha u : \alpha \in \mathbb{R} \}$$

is a one-dimensional subspace of $V$. In fact, all one-dimensional subspaces are of this form.

Now let $T \in \mathcal{L} (V)$. The subspace $U$ is invariant under $T$ if for all $u \in U$ we have $Tu \in U$, in other words if

$$Tu = \lambda u$$

for some $\lambda \in \mathbb{R}$. If this is the case then we say that $\lambda$ is an \textit{eigenvalue} of $T$ and that $u$ is the corresponding \textit{eigenvector}. Observe that if $u$ is an eigenvector of $T$ corresponding to the eigenvalue $\lambda$ then any scalar multiple $\alpha u$ of $u$ is also an eigenvector of $T$ with the eigenvalue $\alpha \lambda$ because $T(\alpha u) = \alpha (Tu) = \alpha \lambda u$. This is another way of looking at the invariance of $U$.

\vskip 3mm
Another way of looking at eigenvalues and eigenvectors is this:  the equation $Tu = \lambda u$ is equivalent to $(T - \lambda I)u = 0$, where $I$ is the identity. This second equation is telling us that $u$ is an eigenvector of $T$ precisely when $u$ is in the null space (or kernel) of $T - \lambda I$. In other words, the set of eigenvectors of $T$ corresponding to $\lambda$ equals the null space of $T - \lambda I$, written $\textrm{null} (T - \lambda I)$! Remember from last chapter that this is a subspace.

This is a revelation. It means that we can determine whether or not $\lambda$ is an eigenvalue of $T$ by checking if $T - \lambda I$ has a non-trivial kernel, i.e. if $\textrm{dim} \:  \textrm{null} (T - \lambda I) > 0$. Equivalently, we can check if $T - \lambda I$ is not injective and therefore not invertible.


Let's look at an example. Suppose that $T \in \mathcal{L} (\mathbb{R}^2)$ is the linear map that we looked at above, with $T((x_1, x_2)^T)) = (3x_1, x_2)^T$. We want to determine if $T$ has any eigenvalues and eigenvectors. To do this, let's go back to the definition. Suppose $T$ does have an eigenvalue and call it $\lambda$. Then we have

\[ 
\begin{pmatrix}
3x_1 \\
x_2
\end{pmatrix}
=
\lambda \begin{pmatrix}
x_1 \\
x_2
\end{pmatrix}
\]

for some non-zero vector, i.e. both $x_1 \neq 0$ and $x_2 \neq 0$. In the first coordinate we have $3x_1 = \lambda x_1$ and in the second coordinate we have $x_2 = \lambda x_2$. Clearly there's no value of $\lambda$ that satisfies both of these equations simultaneously for all values of $x_1, x_2$. But remember that we can choose one of $x_1, x_2$ to be zero (but not both) and go from there. Let's choose $x_2 = 0$. Then the second equation holds no matter what $\lambda$ is and the first equation holds for all $x_1$ when $\lambda = 3$. Therefore \textit{any} non-zero vector of the form $(x_1, 0)^T$ is an eigenvector of $T$ corresponding to the eigenvalue 3. In other words,

$$\textrm{null} (T - 3I) = \{ (x_1, 0)^T : x_1 \in \mathbb{R} \}.$$

But we also could have chosen $x_1 = 0$. In this case the first equation equation holds no matter what $\lambda$ is and the second equation holds for all $x_2$ when $\lambda = 1$. Therefore any non-zero vector of the form $(0, x_2)^T$ is an eigenvector of $T$ corresponding to the eigenvalue 1, i.e.

$$\textrm{null} (T - I) = \{ (0, x_2)^T : x_2 \in \mathbb{R} \}.$$

These are the only eigenvalues of $T$ and their only corresponding sets of eigenvectors. Observe that both $\textrm{null} (T - 3I)$ and $\textrm{null} (T - I)$ are invariant subspaces of dimension one as promised.

\vskip 3mm
One last thing: As we've seen from this example, while a linear map $T$ may have a few eigenvalues, it always has infinitely many eigenvectors corresponding to its eigenvalues. Therefore it's usually convenient to pick the simplest possible eigenvector corresponding to a particular eigenvalue. For example, in the case we just looked at, we would probably say that $(1, 0)^T$ is the eigenvector corresponding to $\lambda = 3$ and $(0, 1)^T$ is the eigenvector corresponding to $\lambda = 1$. Strictly speaking any scalar multiple of those vectors is also an eigenvector, but picking the simplest one makes our lives easier. 

\end{document}