\documentclass[12pt]{article}
\usepackage{mathtools}
\usepackage{amsfonts}
\usepackage{enumerate}
\usepackage{amsthm}
\usepackage{epic}
\usepackage{eepic}
\usepackage{paralist}
\usepackage{graphicx}
\usepackage{algorithm,algorithmic}
\usepackage{tikz}
\usepackage{xcolor,colortbl}
\usepackage{wrapfig}

\title{Chapter 3}
\author{Ben Zelditch}

\begin{document}

\maketitle

\section*{Eigenvalues and Eigenvectors}

In the last chapter we learned all about linear transformations from one vector space to another. Now we're going to focus our attention on the study
of linear operators, which are linear transformations from a vector space to itself. The deepest results in linear algebra are related to linear operators, so this chapter
is going to be a good one. Stay tuned.

\subsection*{Invariant Subspaces}

Before we can think about eigenvalues and eigenvectors, we need a refresher on linear operators. Remember from last chapter that a linear operator $T \in \mathcal{L} (V)$ on a vector space $V$ is a linear map from $V$ to itself. Now imagine that $T$ is some particularly nasty linear map and is complicated to visualize because $V$ is a high dimensional vector space. Remember from Chapter 1 that we could decompose

$$V = U_1 \oplus U_2 \oplus \dots \oplus U_n$$

where $\oplus$ is the direct sum. If we could understand how $T$ behaves on each subspace $U_j$ individually, then we would understand how it behaves on $V$ by linearity. Let $T|_{U_j}$ denote the \textit{restriction} of $T$ to the subspace $U_j$, i.e. we're considering $U_j$ to be the domain of $T$ instead of $V$. Since $U_j$ is smaller than $V$ it makes sense to think that it would easier to deal with $T$ on $U_j$ than on $V$.

But what if $T|_{U_j}$ doesn't map to $U_j$ to itself, i.e. what if there exists some $u \in U_j$ such that $Tu \notin U_j$? In other words, what if $T|_{U_j}$ is not an operator on $U_j$?

Well that would screw things up. Instead we'll only consider decompositions of $V$ such that $T|_{U_j}$ is an operator on $U_j$. This has a name:  we say that a subspace $U \subset V$ is \textit{invariant} under a linear operator $T$ if for all $u \in U$ we have $Tu \in U$, equivalently if $T|_U$ is an operator on $U$. So from now on, we'll only consider direct sum decompositions of invariant subspaces.

Let's drive this point home by looking at a few examples of invariant subspaces. Let $T \in \mathcal{L}(V)$. Then $\{0\}$ and $V$ itself are invariant subspaces (the so called trivial cases). Additionally, $\textrm{null} \: T$ and $\textrm{range} \: T$ are both invariant under $T$ (the proofs for both cases are easy). For a more concrete example, suppose that $T \in \mathcal{L} (\mathbb{R}^2)$ is given by the map

\[ \begin{pmatrix}
x_1 \\
x_2
\end{pmatrix}
%
\mapsto
\begin{pmatrix}
3x_1 \\
x_2
\end{pmatrix}
\]

Then the $x$-axis is invariant under $T$ because if $(x_1, 0)^T$ is a point on the $x$-axis then $T((x_1, 0)^T) = (3x_1, 0)^T$ is also a point on the $x$-axis.

That's about it for invariant subspaces. We'll see why they're so important in the next few sections.


\subsection*{Eigenvalues and Eigenvectors}

Now that we're pretty familiar with invariant subspaces, let's look at a special case of them:  one-dimensional invariant subspaces. Let's first define the one-dimensional subspace.
Let $V$ be a vector space, say $V = \mathbb{R}^n$, and let $u \in V$. Then the set

$$U = \{ \alpha u : \alpha \in \mathbb{R} \}$$

is a one-dimensional subspace of $V$. In fact, all one-dimensional subspaces are of this form.

Now let $T \in \mathcal{L} (V)$. The subspace $U$ is invariant under $T$ if for all $u \in U$ we have $Tu \in U$, in other words if

$$Tu = \lambda u$$

for some $\lambda \in \mathbb{R}$. If this is the case then we say that $\lambda$ is an \textit{eigenvalue} of $T$ and that $u$ is the corresponding \textit{eigenvector}. Observe that if $u$ is an eigenvector of $T$ corresponding to the eigenvalue $\lambda$ then any scalar multiple $\alpha u$ of $u$ is also an eigenvector of $T$ with the eigenvalue $\alpha \lambda$ because $T(\alpha u) = \alpha (Tu) = \alpha \lambda$. This is another way of looking at the invariance of $U$.


\end{document}